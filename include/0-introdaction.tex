\anonsection{Введение}

\textbf{Тема индивидуального задания}: 
Сравнительный анализ инструментов мониторинга в ИТ-системах и их воздействие на рабочие процессы системных администраторов

\textbf{Рабочий план-график практики}:

\begin{longtable}{|l|p{6cm}|p{6cm}|}
    \hline
    № этапа & Наименование этапа                                                                                                                                      & Задание этапа                                                                                                                                                        \\ \hline
    1       & Инструктаж обучающегося                                                                                                                                 & Инструктаж обучающегося по ознакомлению с требованиями охраны труда, техники безопасности, пожарной безопасности, а также правилами внутреннего трудового распорядка \\ \hline
    2       & Изучение и анализ современных инструментов мониторинга, используемых в ИТ-системах.                                                                     & Подготовить обзор существующих инструментов мониторинга                                                                                                              \\ \hline
    3       & Определение основных функций и возможностей каждого инструмента.                                                                                        & Определение основного инструментария.                                                                                                                                \\ \hline
    4       & Оценка эффективности использования каждого инструмента в рабочих процессах системных администраторов.                                                   & Сбор и обработка данных                                                                                                                                              \\ \hline
    5       & Проведение сравнительного анализа между инструментами мониторинга по различным параметрам (функциональность, простота использования, стоимость и т.д.). & Разработка рекомендаций по выбору наиболее подходящего инструмента мониторинга для конкретных задач и условий работы системных администраторов.                      \\ \hline
    6       & Написание и подготовка отчета                                                                                                                           & При написание отчета использовать научные статьи в рецензируемых журналах (не менее 20) и другие материалы.                                                          \\ \hline
\end{longtable}

Цель работы состоит в том, чтобы выявить из множества систем мониторинга подходящую систему для определенных условий.

В ходе выполнения индивидуального задания, проанализированы научные статьи 
на тему сравнения и анализа современных инструментов мониторинга. Проанализированы 
с помощью сайтов документаций об основных функциях каждого инструмент и проведен сравнительный анализ каждого
инструмента, сделан вывод по итогам сравнительного анализа.

\clearpage