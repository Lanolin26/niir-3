\subsection{Оценка эффективности использования каждого инструмента в рабочих процессах системных администраторов}

Сначала необходимо определить основные момента рабочего процесса у системных администраторов. 
Рабочий процесс может варьироваться в зависимости от конкретной организации и должностных обязанностей.
Поиск в информационной поисковой системе Google (www.google.com) по ключевым словам: профессия системный администратор
- выдал различные варианты. Обобщая их, системный администратор может выполнять следующие задачи:
\begin{enumerate}
    \item Мониторинг и обслуживание аппаратного и программного обеспечения
    \item Установка и настройка нового оборудования
    \item Управление учетными записями пользователей
    \item Обеспечение безопасности
    \item Разработка и поддержка политик и процедур
    \item Поддержка пользователей
\end{enumerate}

Использование систем мониторинга позволяет автоматизировать и сделать их наглядными, оперативно оповещающими, следующие процессы:
\begin{itemize}
    \item Мониторинг и обслуживание аппаратного и программного обеспечения
    \item Обеспечение безопасности
    \item Поддержка пользователей
\end{itemize}

Исходя из основных функция инструментов можно сделать вывод о эффективности каждого инструмента. 

Zabbix автоматизирует сбор и агрегацию доступных показателей аппаратного или программного обеспечения. Наглядно демонстрирует 
текущее состояние при помощи визуализации. При помощи реакций на отклонения от нормальных показателей, автоматически 
происходит информирование о проблемах.

ELK автоматизирует сбор и агрегацию показателей и текстовых данных файлов с различных устройств. Гибкая настройка на любой тип показателей.
При помощи технологий машинного обучение и простых проверок происходит информирование о проблемных местах. Сохранение
показателей долгое время способствует анализу безопасности и предсказания других показателей.

Prometheus и Grafana автоматизирует сбор и агрегацию доступных показателей, в большинстве своем численных с различных устройств. 
Наглядная визуализация позволяет понять проблемные места.

Каждый инструмент позволяет экономить время и ресурсы на отслеживание состояние системы (компонента системы, аппаратной системы).

Исходя из личного опыта, например, для подключения с помощью SSH к удаленной виртуальной машины необходимо примерно 5 сек, для поиска и отслеживание 
проблемных показателей от 5 сек до нескольких часов. Системы мониторинга позволяют смотреть на показатели комплексно. Для получения
информации о показателях с систем мониторинга требуется несколько минут. Плюсом систем является хранение историй показателей, что 
сделать вручную достаточно сложно.

Время реагирование на неисправность системы значительно возрастает. Сообщение о неисправности системы будет от конечного потребителя системы,
и оно может достигать большого количества времени. Мониторинг обрабатывает событие о неисправности быстрее и сразу информирует о проблемах.