\subsection{Определение основных функций и возможностей каждого инструмента}

Рассмотрим инструменты Elasticsearch, Zabbix, Prometheus.

Для определения основных функция и возможностей каждого инструмента необходимо обратиться
к официальной документации к каждой системе. 

Для изучения системы Zabbix обратимся к сайту с официальной документацией \href{https://www.zabbix.com/ru}{https://www.zabbix.com/ru}.
В частности, документация на последнюю версию дистрибутива расположена по адресу
 \href{https://www.zabbix.com/documentation/6.0/ru/manual}{https://www.zabbix.com/documentation/6.0/ru/manual}.
В ходе изучения документации были выявлены следующие возможности системы Zabbix:
\begin{itemize}
    \item Сбор данных с различных устройств и типов протоколов.
    \item Гибкие определения порогов.
    \item Множество настроек оповещений.
    \item Построение графиков в режиме реального времени.
    \item Хранение данных истории.
    \item Использование шаблонов.
    \item Сетевое обнаружение.
\end{itemize}


Для изучения системы Elasticsearch и его комбинации в стек технологий ELK (Elasticsearch, Logstash, Kibana) обратимся к сайту с официальной документацией
\href{https://www.elastic.co/elastic-stack}{https://www.elastic.co/elastic-stack}.
В ходе изучения документации были выявлены следующие возможности системы:
\begin{itemize}
    \item Elasticsearch предоставляет мощные возможности быстрого и эффективного полнотекстового поиска, инструмент проводит анализ данных, включая агрегацию, фильтрацию, группировку и подсчеты.
    \item Logstash собирает данные из разных источников.
    \item После сбора данных Logstash может выполнять различные операции над ними. В частности, это фильтрация, структурирование, нормализация и обогащение для дальнейшего анализа.
    \item Kibana дает возможность создавать различные типы визуализаций.
    \item ELK стек легко масштабируется.
    \item EQL, гибкий язык запросов.
\end{itemize}



Для изучения системы Prometheus и его необходимых компонентов Grafana и Node exporter обратимся к сайту с официальной документацией 
\href{https://prometheus.io/}{https://prometheus.io/} и \href{https://grafana.com/}{https://grafana.com/}.
В ходе изучения документации были выявлены следующие возможности системы:
\begin{itemize}
    \item Изначально создавалась под среду Kubernetes, но можно использовать на сервера и приложения.
    \item Многомерная модель данных с данными временных рядов.
    \item PromQL, гибкий язык запросов для использования этой размерности.
    \item Prometheus не имеет возможности визуализировать собранные данные. Поэтому часто его устанавливают в купе с Grafana
    \item Grafana предоставляет список панельных модулей, позволяющих создавать красивые визуализации
\end{itemize}






