\begingroup
\renewcommand{\section}[2]{\anonsection{Список использованных источников}}

% \bibliographystyle{biblatex-gost}
\bibliography{include/biblist}

% \begin{thebibliography}{1}
%     \def\selectlanguageifdefined#1{
%         \expandafter\ifx\csname date#1\endcsname\relax
%         \else\language\csname l@#1\endcsname\fi}
%     \ifx\undefined\url\def\url#1{{\small #1}}\else\fi
%     \ifx\undefined\BibUrl\def\BibUrl#1{\url{#1}}\else\fi
%     \ifx\undefined\BibAnnote\long\def\BibAnnote#1{}\else\fi
%     \ifx\undefined\BibEmph\def\BibEmph#1{\emph{#1}}\else\fi
    
%     \bibitem{markina}
%     \selectlanguageifdefined{english}
%     \BibEmph{Маркина~Т.А.~Пенской~А.В.~Штенников~Д.Г.}
%     Производственная практика магистрантов:
%     организация и проведение. ---
%     \newblock СПб:Университет ИТМО, 2020. ---
%     \newblock P.~50.
    
% \end{thebibliography}


%\hypersetup{ urlcolor=black }               % Ссылки делаем чёрными
%\providecommand*{\BibDash}{}                % В стилях ugost2008 отключаем использование тире как разделителя
% \urlstyle{rm}                               % ссылки URL обычным шрифтом
% \ifdefmacro{\microtypesetup}{\microtypesetup{protrusion=false}}{} % не рекомендуется применять пакет микротипографики к автоматически генерируемому списку литературы
% \insertbibliofull                           % Подключаем Bib-базы: все статьи единым списком
% Режим с подсписками
%\insertbiblioexternal                      % Подключаем Bib-базы: статьи, не являющиеся статьями автора по теме диссертации
% Для вывода выберите и расскомментируйте одно из двух
%\insertbiblioauthor                        % Подключаем Bib-базы: работы автора единым списком 
%\insertbiblioauthorgrouped                 % Подключаем Bib-базы: работы автора сгруппированные (ВАК, WoS, Scopus и т.д.)
% \ifdefmacro{\microtypesetup}{\microtypesetup{protrusion=true}}{}
% \urlstyle{tt}                               % возвращаем установки шрифта ссылок URL
%\hypersetup{ urlcolor={urlcolor} }          % Восстанавливаем цвет ссылок

% \bibliographystyle{biblatex-gost}
% \bibliography{include/biblist}

% \renewcommand{\section}[2]{\anonsection{Список использованных источников}}
% \begin{thebibliography}{00}
% \bibitem{phd}
% Шматков, В.Н.,  
% 2021. Архитектура и организация граничных вычислений для виртуального кластера на основе компьютеров с ограниченными вычислительными ресурсами, 
% PhD thesis, 
% ИТМО, 
% Санкт-Петербург.

% \bibitem{dist-data}
% Большой ликбез: распределённые системы хранения данных в практической 
% привязке для админов среднего и крупного бизнеса // Хабр.
% URL: \href{https://bit.ly/3w9pz85}{https://bit.ly/3w9pz85}
% (дата обращения: 18.04.2022)
%https://habr.com/ru/company/croc/blog/272795/

% \end{thebibliography}
\endgroup


\clearpage