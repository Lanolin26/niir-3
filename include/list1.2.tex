В итоге в поисковой системе <<КиберЛенинка>> были найдены следующие статьи, которые подходили по тематике.

Статья <<АНАЛИЗ СУЩЕСТВУЮЩИХ СРЕДСТВ ПО РЕШЕНИЮ ЗАДАЧИ МОНИТОРИНГА СЕТЕВОЙ ИНФРАСТРУКТУРЫ ПРЕДПРИЯТИЯ>> \cite{instrumentmonitoring}
представляет обзор средств по решению задачи мониторинга сетевой инфраструктуры. 
Рассмотрены зарубежные решения (Zabbix и Nagios), представлены их характеристики, описаны ключевые возможности, а также их недостатки. 

В статье <<Сравнительный анализ популярных систем мониторинга сетевого оборудования, распространяемых по лицензии GPL>> 
\cite{sravnitelnyy-analiz-populyarnyh-sistem-monitoringa-setevogo-oborudovaniya-rasprostranyaemyh-po-litsenzii-gpl}
представлены наиболее популярные системы мониторинга сетевого оборудования, распространяемые по лицензии GPL Cacti, Nagios, Zabbix. 
В ней выявлены сходства и различия между ними, а так же описаны архитектура и основные компоненты системы мониторинга.

Статья <<Системы мониторинга оборудования>> \cite{sistemy-monitoringa-oborudovaniya}
рассматривает организацию сетевого мониторинга оборудования с использованием Zabbix. 
Представлены ключевые моменты сетевого мониторинга.

В статье <<ZABBIX ДЛЯ МОНИТОРИНГА В IT-ИНФРАСТРУКТУРЕ>> 
\cite{zabbix-dlya-monitoringa-v-it-infrastrukture}
рассматривается система Zabbix, возможные параметры наблюдения. Так же описываются преимущества Zabbix среди других систем мониторинга.

Статья <<ОПТИМАЛЬНЫЙ ВЫБОР СИСТЕМЫ МОНИТОРИНГА ДЛЯ РАЗЛИЧНЫХ ТИПОВ ИТ-ИНФРАСТРУКТУР>>
\cite{optimalnyy-vybor-sistemy-monitoringa-dlya-razlichnyh-tipov-it-infrastruktur}
описывает различные системы мониторинга. Проведен сравнительный анализ систем мониторинга для различных типов ИТ-инфраструктур.

В статье <<ОБЗОР УТИЛИТ МОНИТОРИНГА LINUX СИСТЕМ>> \cite{obzor-utilit-monitoringa-linux-sistem}
произведен обзор утилит мониторинга Linux серверов. 
Изучена необходимость в мониторинге серверов и рассмотрены типы мониторинга, такие как системный мониторинг, сетевой мониторинг, облачный мониторинг.
Рассмотрены плюсы и минусы различных систем мониторинга а также ПО.

Статья <<Системы мониторинга. Обзор и сравнение>> \cite{sistemy-monitoringa-obzor-i-sravnenie}
рассматривается понятие системы мониторинга ИТ, а также, где они используются. В данной статье рассмотрены главные принципы администрирования систем,
произведен сравнительный анализ нескольких программ, реализующих мониторинг

Статья <<Мониторинг виртуальной вычислительной системы>> \cite{monitoring-virtualnoy-vychislitelnoy-sistemy}
посвящена вопросам мониторинга вычислительных систем, использующих виртуализации. Описаны основные принципы мониторинга виртуальных систем.
Так же описаны принципы по созданию мониторинга с нуля.

Статья <<РАЗРАБОТКА СИСТЕМЫ МОНИТОРИНГА ДЛЯ СЕРВЕРНОГО ПРИЛОЖЕНИЯ>> \cite{razrabotka-sistemy-monitoringa-dlya-servernogo-prilozheniya}
описывает инструменты мониторинга и пример внедрения в организацию. Так же описаны собираемые метрики, архитектуры.

В статье <<ПРИМЕНЕНИЕ СТЕКА ТЕХНОЛОГИЙ ELK ДЛЯ СБОРА И АНАЛИЗА СИСТЕМНЫХ ЖУРНАЛОВ СОБЫТИЙ>> \cite{primenenie-steka-tehnologiy-elk-dlya-sbora-i-analiza-sistemnyh-zhurnalov-sobytiy}
рассматривается внедрение системы централизованного сбора и анализа системных журналов событий. В качестве системы был взят стек технологий Elasticsearch, Logstash, Kibana (ELK).,

Статья <<МОНИТОРИНГ И ПОИСК НЕИСПРАВНОСТЕЙ В РАСПРЕДЕЛЁННЫХ ВЫСОКОНАГРУЖЕННЫХ СИСТЕМАХ>> \cite{monitoring-i-poisk-neispravnostey-v-raspredelyonnyh-vysokonagruzhennyh-sistemah}
описывает наиболее популярные инструменты, которые используются в при разработке высоконапряженных систем. 
Так же разбираются аналоги используемых инструментов