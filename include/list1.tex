\subsection{Изучение и анализ современных инструментов мониторинга, используемых в ИТ-системах}

Для поиска инструментов мониторинга, а так же их сравнительный анализ и способы внедрения 
были использованы следующие информационные ресурсы: <<КиберЛенинка>> (cyberleninka.ru), Google Academy (scholar.google.ru).

В поисковой системе Google Academy (scholar.google.ru) для поиска использовались ключевые слова: 
monitoring, it, system administrator, administrator. Так же использовался фильтр по дате публикации: от 2010 до 2023.
Для ограничения результатов и фильтрации, были убраны из поиска
все статьи, где упоминались слова: iot, medicine. К ключевому слову medicine было много статей, не относящиеся к теме поиска.
К ключевому слову iot были найдены огромное количество статей, где не был найдено полезного по теме.

В итоге в поисковой системе Google Academy были найдены следующие статьи, которые подходили по тематике.

Статья <<Systems Monitoring and Big Data Analysis Using the Elasticsearch System>> \cite{8745151} описывает построение
системы мониторинга на базе технологии Elasticsearch. В ней описывается структура системы и результаты использования.

В статье <<Monitoring in a DevOps World: Perfect should never be the enemy of better>> \cite{10.1145/3178368.3178371}
описывается, что такое мониторинг и как его можно построить с учетом DevOps практик.

Книга <<Nagios, 2nd Edition: System and Network Monitoring>> \cite{barth2008nagios} рассказывает о системе мониторинга 
Nagios, ее настройки и варианты использования.

Статья <<Monitoring distributed systems>> \cite{244791} описывает общий вид систем мониторинга, а так же данные,
которые необходимы для корректного анализа.


В поисковой системе <<КиберЛенинка>> (cyberleninka.ru) для поиска использовались ключевые слова: 
мониторинг, ИТ, системное администрирование, обзор инструментов. Так же использовался фильтр по дате публикации: 
от 2010 до 2023.Д ля ограничения результатов и фильтрации, был использован фильтр по терму OECD
<<Компьютерные и информационные науки>>.

В итоге в поисковой системе <<КиберЛенинка>> были найдены следующие статьи, которые подходили по тематике.

Статья <<АНАЛИЗ СУЩЕСТВУЮЩИХ СРЕДСТВ ПО РЕШЕНИЮ ЗАДАЧИ МОНИТОРИНГА СЕТЕВОЙ ИНФРАСТРУКТУРЫ ПРЕДПРИЯТИЯ>> \cite{instrumentmonitoring}
представляет обзор средств по решению задачи мониторинга сетевой инфраструктуры. 
Рассмотрены зарубежные решения (Zabbix и Nagios), представлены их характеристики, описаны ключевые возможности, а также их недостатки. 

В статье <<Сравнительный анализ популярных систем мониторинга сетевого оборудования, распространяемых по лицензии GPL>> 
\cite{sravnitelnyy-analiz-populyarnyh-sistem-monitoringa-setevogo-oborudovaniya-rasprostranyaemyh-po-litsenzii-gpl}
представлены наиболее популярные системы мониторинга сетевого оборудования, распространяемые по лицензии GPL Cacti, Nagios, Zabbix. 
В ней выявлены сходства и различия между ними, а так же описаны архитектура и основные компоненты системы мониторинга.

Статья <<Системы мониторинга оборудования>> \cite{sistemy-monitoringa-oborudovaniya}
рассматривает организацию сетевого мониторинга оборудования с использованием Zabbix. 
Представлены ключевые моменты сетевого мониторинга.

В статье <<ZABBIX ДЛЯ МОНИТОРИНГА В IT-ИНФРАСТРУКТУРЕ>> 
\cite{zabbix-dlya-monitoringa-v-it-infrastrukture}
рассматривается система Zabbix, возможные параметры наблюдения. Так же описываются преимущества Zabbix среди других систем мониторинга.

Статья <<ОПТИМАЛЬНЫЙ ВЫБОР СИСТЕМЫ МОНИТОРИНГА ДЛЯ РАЗЛИЧНЫХ ТИПОВ ИТ-ИНФРАСТРУКТУР>>
\cite{optimalnyy-vybor-sistemy-monitoringa-dlya-razlichnyh-tipov-it-infrastruktur}
описывает различные системы мониторинга. Проведен сравнительный анализ систем мониторинга для различных типов ИТ-инфраструктур.

В статье <<ОБЗОР УТИЛИТ МОНИТОРИНГА LINUX СИСТЕМ>> \cite{obzor-utilit-monitoringa-linux-sistem}
произведен обзор утилит мониторинга Linux серверов. 
Изучена необходимость в мониторинге серверов и рассмотрены типы мониторинга, такие как системный мониторинг, сетевой мониторинг, облачный мониторинг.
Рассмотрены плюсы и минусы различных систем мониторинга а также ПО.

Статья <<Системы мониторинга. Обзор и сравнение>> \cite{sistemy-monitoringa-obzor-i-sravnenie}
рассматривается понятие системы мониторинга ИТ, а также, где они используются. В данной статье рассмотрены главные принципы администрирования систем,
произведен сравнительный анализ нескольких программ, реализующих мониторинг

Статья <<Мониторинг виртуальной вычислительной системы>> \cite{monitoring-virtualnoy-vychislitelnoy-sistemy}
посвящена вопросам мониторинга вычислительных систем, использующих виртуализации. Описаны основные принципы мониторинга виртуальных систем.
Так же описаны принципы по созданию мониторинга с нуля.

Статья <<РАЗРАБОТКА СИСТЕМЫ МОНИТОРИНГА ДЛЯ СЕРВЕРНОГО ПРИЛОЖЕНИЯ>> \cite{razrabotka-sistemy-monitoringa-dlya-servernogo-prilozheniya}
описывает инструменты мониторинга и пример внедрения в организацию. Так же описаны собираемые метрики, архитектуры.

В статье <<ПРИМЕНЕНИЕ СТЕКА ТЕХНОЛОГИЙ ELK ДЛЯ СБОРА И АНАЛИЗА СИСТЕМНЫХ ЖУРНАЛОВ СОБЫТИЙ>> \cite{primenenie-steka-tehnologiy-elk-dlya-sbora-i-analiza-sistemnyh-zhurnalov-sobytiy}
рассматривается внедрение системы централизованного сбора и анализа системных журналов событий. В качестве системы был взят стек технологий Elasticsearch, Logstash, Kibana (ELK).,

Статья <<МОНИТОРИНГ И ПОИСК НЕИСПРАВНОСТЕЙ В РАСПРЕДЕЛЁННЫХ ВЫСОКОНАГРУЖЕННЫХ СИСТЕМАХ>> \cite{monitoring-i-poisk-neispravnostey-v-raspredelyonnyh-vysokonagruzhennyh-sistemah}
описывает наиболее популярные инструменты, которые используются в при разработке высоконапряженных систем. 
Так же разбираются аналоги используемых инструментов

Так же был произведен поиск в информационной системе <<Харб>> (habr.com). В этой системе описываются
методики применение инструментов, их сравнительный анализ и способы их настройки.
Для поиска применялись следующие ключевые слова: мониторинг систем, системный администратор, обзор 
инструментов, сравнение инструментов. 

В информационной системе <<Харб> были найдены следующие статьи.

В статье <<Основы мониторинга (обзор Prometheus и Grafana)>> \cite{habr-709204}
описаны возможные системы мониторинга, их базовые возможности и связки друг с другом.
Так же приведен пример использования базовой версии стека технологий PAVG (Prometheus, Alertmanager, VictoriaMetrics, Grafana).

Статья <<Сравниваем инструменты мониторинга IT-инфраструктуры Zabbix, Icinga, Prometheus>> \cite{habr-705464}
описывает технологии мониторинга, приводятся примеры их внутренних подсистем для хранения, интерфейс отображения
и возможные оповещения.

В статье <<Организация системы мониторинга>> \cite{habr-350200} показаны системы мониторинга, их 
описание. Так же рассказано про проектирование, метрики и оповещения. 

После просмотра и анализа статей, были выделены несколько систем мониторинга: Elasticsearch, Zabbix, Prometheus.
Они будут исследоваться более подробное.