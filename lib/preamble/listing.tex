%---------------------------Listing------------------------------------%
\usepackage{listings}
\DeclareCaptionFormat{listing}{
    \parbox{\linewidth}{
        \colorbox{сaptionbk}{
            \parbox{\linewidth}{
                \small#1#2#3
            }
        }
        \vskip-4pt
    }
}
% \captionsetup[lstlisting]{format=listing, labelfont=white, textfont=white}
\lstset{% Собственно настройки вида листинга
    inputencoding=utf8, 
    extendedchars=\true, 
    language=Java,
    keepspaces=true,            % поддержка кириллицы  и пробелов в комментариях
    basicstyle=\footnotesize,,  % размер и начертание шрифта для подсветки кода
    numbers=left,               % где поставить нумерацию строк (слева\справа)
    numberstyle=\tiny,          % размер шрифта для номеров строк
    stepnumber=1,               % размер шага между двумя номерами строк
    numbersep=5pt,              % как далеко отстоят номера строк от подсвечиваемого кода
    showspaces=false,           % показывать или нет пробелы специальными отступами
    showstringspaces=false,     % показывать или нет пробелы в строках
    showtabs=false,             % показывать или нет табуляцию в строках
    frame=single,               % рисовать рамку вокруг кода
    tabsize=2,                  % размер табуляции по умолчанию равен 2 пробелам
    captionpos=t,               % позиция заголовка вверху [t] или внизу [b] 
    breaklines=true,            % автоматически переносить строки (да\нет)
    breakatwhitespace=false,    % переносить строки только если есть пробел
    % escapeinside={\\*}{*)},      % если нужно добавить комментарии в коде
    literate={--}{{-{}-}}2,     % Корректно отображать двойной дефис
    literate={---}{{-{}-{}-}}3,  % Корректно отображать тройной дефис
    escapechar=~,
    frame=lrb,
    xleftmargin=\fboxsep,
    xrightmargin=-\fboxsep,
    otherkeywords={!,!=,~,\$,*,\&,\%/\%,\%*\%,\%\%,<-,<<-,\_,/},
}
%----------------------------------------------------------------------%

% \newcommand\YAMLcolonstyle{\color{red}\mdseries}
% \newcommand\YAMLkeystyle{\color{black}\bfseries}
% \newcommand\YAMLvaluestyle{\color{blue}\mdseries}
\newcommand\YAMLcolonstyle{\color{red}}
\newcommand\YAMLkeystyle{\color{black}\footnotesize}
\newcommand\YAMLvaluestyle{\color{blue}}

\makeatletter

% here is a macro expanding to the name of the language
% (handy if you decide to change it further down the road)
\newcommand\language@yaml{yaml}

\expandafter\expandafter\expandafter\lstdefinelanguage
\expandafter{\language@yaml}{
    keywords={true,false,null,y,n},
    keywordstyle=\color{darkgray}\bfseries,
    basicstyle=\YAMLkeystyle,                                 % assuming a key comes first
    sensitive=false,
    comment=[l]{\#},
    morecomment=[s]{/*}{*/},
    commentstyle=\color{purple}\ttfamily,
    stringstyle=\YAMLvaluestyle\ttfamily,
    moredelim=[l][\color{orange}]{\&},
    moredelim=[l][\color{magenta}]{*},
    moredelim=**[il][\YAMLcolonstyle{:}\YAMLvaluestyle]{:},   % switch to value style at :
    morestring=[b]',
    morestring=[b]",
    literate = {---}{{\ProcessThreeDashes}}3
            {>}{{\textcolor{red}\textgreater}}1     
            {|}{{\textcolor{red}\textbar}}1 
            {\ -\ }{{\mdseries\ -\ }}3,
}

% switch to key style at EOL
\lst@AddToHook{EveryLine}{\ifx\lst@language\language@yaml\YAMLkeystyle\fi}
\makeatother

\newcommand\ProcessThreeDashes{\llap{\color{cyan}\mdseries-{-}-}}
